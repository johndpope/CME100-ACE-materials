\documentclass[letterpaper, 11pt]{article}
\usepackage{comment} % enables the use of multi-line comments (\ifx \fi) 
\usepackage{lipsum} %This package just generates Lorem Ipsum filler text. 
\usepackage{fullpage} % changes the margin

\usepackage{fancyhdr} % Required for custom headers
\usepackage{lastpage} % Required to determine the last page for the footer
\usepackage{extramarks} % Required for headers and footers
\usepackage{mdframed}
\usepackage{caption}
\usepackage{subcaption}
\usepackage{float}
\usepackage{array}
\usepackage{soul}
\usepackage{amsmath}
\usepackage{mathtools}
\usepackage{graphicx} % Required to insert images
\usepackage{multicol}
\usepackage{enumitem}
\usepackage{amssymb,bm}
\usepackage{verbatim,eufrak,hyperref,bbm}
\usepackage{titlesec}

%%%%% TEMPLATE-SPECIFIC FORMATTING %%%%%
%\usepackage{fourier}
\usepackage[adobe-utopia]{mathdesign}
\titleformat{\section}
  {\normalfont\fontsize{12}{15}\bfseries}{\thesection.}{1em}{}
  \titleformat{\subsection}[runin]{\normalfont}{\thesubsection}{3pt}{}
%\usepackage[T1]{fontenc}

%----------------------------------------------------------------------------------------
%	NAME AND CLASS SECTION
%----------------------------------------------------------------------------------------

\newcommand{\hmwkTitle}{Week\ 2\ Worksheet} % Assignment title
\newcommand{\hmwkClass}{CME\ 100\ ACE} % Course/class
\newcommand{\hmwkAuthorName}{Timothy Anderson} % Your name
\newcommand{\hmwkAuthorEmail}{timmya@stanford.edu} % Your email

% Set up the header and footer
\pagestyle{fancy}
\lhead{} % Top left header
\chead{} % Top center header
\rhead{} % Top right header
\lfoot{\hmwkClass\ : \hmwkTitle} % Bottom left footer
\cfoot{Page\ \thepage\ of\ \pageref{LastPage}} % Bottom center footer
\rfoot{\hmwkAuthorName} % Bottom right footer
\renewcommand\headrulewidth{0pt} % Size of the header rule
\renewcommand\footrulewidth{0.4pt} % Size of the footer rule


% Math commands
\DeclareMathOperator*{\argmin}{arg\,min}
\DeclareMathOperator*{\argmax}{arg\,max}
\DeclareMathOperator*{\proj}{proj}
\allowdisplaybreaks

% Margins
\topmargin=-0.45in
\evensidemargin=0in
\oddsidemargin=0in
\textwidth=6.5in
\textheight=9.0in
\headsep=0.25in 

\setlength{\parindent}{0pt} % Set indent to zero

\begin{document}
%Header-Make sure you update this information!!!!


%\thispagestyle{empty}
\noindent
\normalsize 
%\hmwkAuthorName 
\hmwkClass \hfill April\ 10,\ 2017\\
%\hmwkAuthorEmail \\

\begin{center} \Large \textbf{\hmwkTitle} \end{center}


\section{Unit Vectors}
\subsection{} Describe conceptually what a unit vector is. 

\subsection{} Compute the unit vectors for the following vectors.
\begin{enumerate}[label=(\alph*)]
\item $\vec v = \langle 1,1,1 \rangle$

\item $\vec w = \langle 0,-1,-1 \rangle$

\item $\vec u = \langle 10,8,-7 \rangle$

\end{enumerate}

\subsection{} The unit vector from point $A = (0,2,3)$ to $B = (1,6,-2)$.

\subsection{} If $\vec v$ and $\vec w$ are orthogonal, will their unit vectors also be orthogonal? Why? 

\section{Vector Operations}
Compute the following.
\begin{enumerate}[label=(\alph*)]
\item[] \[ \vec v = \langle 1,2,5 \rangle, \quad \vec w = \langle 3, -4, 2 \rangle \]

\item $2 \vec v  - \vec w$

\item $(2\vec v) \cdot \vec w$

\item The unit vector of $\vec v \times \vec w$

\item[] \[ \vec v = \langle 1,0,0 \rangle, \quad \vec w = \langle \sqrt{3}, \sqrt{3},\sqrt{3} \rangle \]

\item $\vec v \cdot \left( \frac{1}{\sqrt{3}} \vec w \right)$

\item $\vec w \cdot \vec w$

\item The angle between $\vec v$ and $\vec w$. 


\end{enumerate}

\section{Projections}
Compute the following.
\begin{enumerate}[label=(\alph*)]
\item[] \[ \vec v = \langle 1,2,5 \rangle, \quad \vec w = \langle 3, -4, 2 \rangle \]

\item $\proj_{\vec v} \vec w$

\item $\proj_{\vec w} \vec v$

\item[] \[ \vec v = \langle 1,0,0 \rangle, \quad \vec w = \langle \sqrt{3}, \sqrt{3},\sqrt{3} \rangle \]

\item $\proj_{\vec v} \vec w$

\item $\proj_{\vec w} \vec v$

\end{enumerate}

\section{Lines and Planes}
\subsection{} Compute the area enclosed by the parallelogram defined by:
\[ A(0,0) \quad B(7,3) \quad C(9,8) \quad D(2,5) \]

\subsection{} Compute the area enclosed by the triangle defined by:
\[ A(0, 0)\quad B(-2,3)\quad C(3, 1) \]

\subsection{} Find the equation for the line through $(1,2,1)$ in the direction of $\vec v = \langle 0, 1, 0 \rangle$

\subsection{} Find the equation for the plane through $(1,2,1)$ with normal $\vec n = \langle -1, 0, 1 \rangle$


\section{Vector-valued functions}
Compute the velocity and acceleration vectors of the following. In (c), also compute the tangent vector at the given point.
\begin{enumerate}[label = (\alph*)]
\item $\vec r (t) = (1 + t) \vec i + \frac{t^2}{\sqrt{2}} \vec j + \frac{t^3}{3} \vec k$

\item $\vec r(t) = \sec(t) \vec i + \tan (t) \vec j + t\vec k$

\item $\vec r(t) =  \ln(t) \vec i+\frac{t - 1}{t + 2} \vec j +t\ln(t) \vec k$, and $t_0 = 1$

\end{enumerate}


%\section{Partial Derivatives}
%\subsection{} Compute the following partial derivatives.
%\begin{enumerate}[label=(\alph*)]
%\item
%
%
%\item
%
%
%\item
%
%
%\item
%
%
%\end{enumerate}
%
%
%\subsection{Application} Suppose we have a vector $\vec \Delta$ such that 
%\[ \vec \Delta = \left[ \begin{array}{c} \frac{\partial}{\partial x} \\ \frac{\partial}{\partial y} \\ \frac{\partial}{\partial z} \end{array} \right] \]
%That is, $\vec \Delta $ when multiplied by a function will produce a vector of the infinitesimal change in that function in each direction at a given point $(x,y,z)$. We call this the \textit{gradient vector}, which you will learn about later in this course. (Why would something like this be useful?)


\end{document}