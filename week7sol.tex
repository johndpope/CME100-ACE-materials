\documentclass[letterpaper, 11pt]{article}
\usepackage{comment} % enables the use of multi-line comments (\ifx \fi) 
\usepackage{lipsum} %This package just generates Lorem Ipsum filler text. 
\usepackage{fullpage} % changes the margin

\usepackage{fancyhdr} % Required for custom headers
\usepackage{lastpage} % Required to determine the last page for the footer
\usepackage{extramarks} % Required for headers and footers
\usepackage{mdframed}
\usepackage{caption}
\usepackage{subcaption}
\usepackage{float}
\usepackage{array}
\usepackage{soul}
\usepackage{amsmath}
\usepackage{graphicx} % Required to insert images
\usepackage{multicol}
\usepackage{enumitem}
\usepackage{amssymb,bm}
\usepackage{verbatim,eufrak,hyperref,bbm}
\usepackage{titlesec}

%%%%% TEMPLATE-SPECIFIC FORMATTING %%%%%
%\usepackage{fourier}
\usepackage[adobe-utopia]{mathdesign}
\titleformat{\section}
  {\normalfont\fontsize{12}{15}\bfseries}{\thesection.}{1em}{}
  \titleformat{\subsection}[runin]{\normalfont}{\thesubsection}{3pt}{}
%\usepackage[T1]{fontenc}

%----------------------------------------------------------------------------------------
%	NAME AND CLASS SECTION
%----------------------------------------------------------------------------------------

\newcommand{\hmwkTitle}{Week\ 7\ Solutions} % Assignment title
\newcommand{\hmwkClass}{CME\ 100\ ACE} % Course/class
\newcommand{\hmwkAuthorName}{Timothy Anderson} % Your name
\newcommand{\hmwkAuthorEmail}{timmya@stanford.edu} % Your email

% Set up the header and footer
\pagestyle{fancy}
\lhead{} % Top left header
\chead{} % Top center header
\rhead{} % Top right header
\lfoot{\hmwkClass\ : \hmwkTitle} % Bottom left footer
\cfoot{Page\ \thepage\ of\ \pageref{LastPage}} % Bottom center footer
\rfoot{\hmwkAuthorName} % Bottom right footer
\renewcommand\headrulewidth{0pt} % Size of the header rule
\renewcommand\footrulewidth{0.4pt} % Size of the footer rule


% Math commands
\DeclareMathOperator*{\argmin}{arg\,min}
\DeclareMathOperator*{\argmax}{arg\,max}
\allowdisplaybreaks

% Margins
\topmargin=-0.45in
\evensidemargin=0in
\oddsidemargin=0in
\textwidth=6.5in
\textheight=9.0in
\headsep=0.25in 

\setlength{\parindent}{0pt} % Set indent to zero

\begin{document}

%\thispagestyle{empty}
\noindent
\normalsize 
%\hmwkAuthorName 
\hmwkClass \hfill May\ 15,\ 2017\\
%\hmwkAuthorEmail \\

\begin{center} \Large \textbf{\hmwkTitle} \end{center}

\section{Polar Coordinates}
Evaluate the following integrals:
\begin{enumerate}[label=(\alph*)]
% TC 15.4 #16
\item $\int_{\sqrt{2}}^2 \int_{\sqrt{ 4- y^2}}^y dx dy$
\par \textbf{Solution:} The main difficulty with using polar coordinates (or any of the other coordinate systems we explore later in this class) is converting between the coordinate systems. From there, evaluating the integral is actually very straightforward. 
\par We can transform the integrand through the substitutions:
\begin{gather*}
dx dy = r dr d \theta \\
x = r \cos \theta\\
y = r \sin \theta
\end{gather*}
For the bounds of integration, it is usually most helpful to draw out the region over which we are integrating, then visually convert this to polar coordinates. After drawing the region and determining the region in polar coordinates, we can compute the integral:
\begin{align*}
\int_{\sqrt{2}}^2 \int_{\sqrt{ 4- y^2}}^y dx dy &= \int_{\pi/4}^{\pi/2} \int_{2}^{2 \csc \theta} r dr d\theta \\
&=  \int_0^\pi \frac{1}{2}\left( 4\csc^2 \theta - 4\right)d\theta \\
&= 2 \int_{\pi/4}^{\pi/2} ( \csc^2 \theta - 1)d\theta \\
&= 2\left[-\cot \theta - \theta   \right]_{\pi/4}^{\pi/2} \\
&= 2\left( - \frac{\pi}{2} + 1 + \frac{\pi}{4} \right)\\
&= 2 - \frac{\pi}{2} \quad\blacksquare 
\end{align*}
\textit{Note:} you need to become very familiar with your trigonometric identities. The remainder of CME 100 will rely very heavily on using these identities or similar substitutions. 

% TC 15.4 #22
\item $ \int_1^2 \int_0^{\sqrt{2x - x^2}} \frac{1}{(x^2 + y^2)^2} dy dx$
\par \textbf{Solution:} The main difficulty here is converting the bounds of integration. Notice that the region of interest is a quarter circle centered at $(1,0)$. 
\begin{align*}
\int_1^2 \int_0^{\sqrt{2x - x^2}} \frac{1}{(x^2 + y^2)^2} dy dx &=  \int_0^{\pi/4} \int_{\sec \theta}^{2\cos\theta}\frac{1}{r^4} rdr d\theta \\
&= \int_0^{\pi/4} \int_{\sec \theta}^{2\cos\theta}\frac{1}{r^3}dr d\theta \\
&= \int_0^{\pi/4} \left[-\frac{1}{2r^2} \right]_{\sec \theta}^{2\cos\theta} d\theta \\
&= \int_0^{\pi/4} \left(-\frac{1}{8} \sec^2 \theta + \frac{1}{2}\cos^2 \theta\right) d\theta \\
&= \left[ \frac{1}{4} \theta + \frac{1}{8} \sin 2 \theta - \frac{1}{8} \tan \theta \right]_0^{\pi/4} \\
&= \frac{\pi}{16} +\frac{1}{8} - \frac{1}{8} = \frac{\pi}{16} \quad\blacksquare 
\end{align*}

% TC 15.4 #18
\item $ \int_{-1}^1 \int_{-\sqrt{1 - x^2}}^{\sqrt{1 - x^2}} \frac{2}{(1 + x^2 + y^2)^2}dy dx$
\par \textbf{Solution:} We are integrating over a circle of radius 1, so the bounds are very easy to set up in this case:
\begin{align*}
\int_{-1}^1 \int_{-\sqrt{1 - x^2}}^{\sqrt{1 - x^2}} \frac{2}{(1 + x^2 + y^2)^2}dy dx &= \int_0^{2\pi} \int_0^1 \frac{2}{(1 + r^2)^2} r dr d\theta \\
&= \int_0^{2\pi} \frac{1}{2} d \theta \\
&= \pi \quad \blacksquare
\end{align*}

\end{enumerate}

\section{Triple Integrals}
% TC 15.5 #18
\subsection{} Evaluate the integral:
\[ \int_0^1 \int_1^{\sqrt{e}} \int_1^e se^s \ln r \frac{(\ln t)^2}{t} dt dr ds \]
\par \textbf{Solution:} The procedure for evaluating triple integrals is nearly identical to evaluating double integrals; in triple integrals, we just have one more variable to worry about and keep track of. The main challenge in triple integrals is learning to spot substitutions and applying identities to make the integral computable. 
\begin{align*}
\int_0^1 \int_1^{\sqrt{e}} \int_1^e se^s \ln r \frac{(\ln t)^2}{t} dt dr ds &= \int_0^1 \int_1^{\sqrt{e}} \int_0^1 se^s \ln r u^2 du dr ds \\
&= \frac{1}{3} \int_0^1 \int_1^{\sqrt{e}} se^s \ln r dr ds \\
&= \frac{1}{3} \int_0^1 \left[ x(\ln x -1 )\right]_1^{\sqrt{e}} se^s ds  \\
&= \frac{1}{3}\left( 1 - \frac{\sqrt{e}}{2}\right) \int_0^1 se^s ds \\
&= \frac{1}{3}\left( 1 - \frac{\sqrt{e}}{2}\right) \left[ e^s(s-1) \right]_0^1 \\
&=  \frac{2 - \sqrt{e}}{6} \quad\blacksquare 
\end{align*}


\subsection{} Evaluate the integral:
\[ \int_0^{\sqrt{2}} \int_0^{3y} \int_{x^2 + 3y^2}^{8-x^2 - y^2} dz dx dy \]
\par \textbf{Solution:} 
\begin{align*}
\int_0^{\sqrt{2}} \int_0^{3y} \int_{x^2 + 3y^2}^{8-x^2 - y^2} dz dx dy &= \int_0^{\sqrt{2}} \int_0^{3y} (8 - 2x^2 - 4y^2) dx dy \\
&= \int_0^{\sqrt{2}} \left[ 8x - \frac{2}{3} x^3 - 4xy^2 \right]_0^{3y} dy \\
&= \int_0^{\sqrt{2}}(24y - 30y^3 ) dy \\
&= \left[12 y^2 - \frac{15}{2} y^4 \right]_0^{\sqrt{2}} \\
&= 12(2) - \frac{15}{2}(4) = -6 \quad\blacksquare 
\end{align*}


% TC 15.7 #6
\section{Look Ahead: Cylindrical Coordinates}
Evaluate the following integral:
\[ \int_0^{2 \pi} \int_0^1 \int_{-1/2}^{1/2} (r^2 \sin^2 \theta + z^2) dz r dr d\theta \]
\par \textbf{Solution:} We treat cylindrical coordinates exactly like regular triple integrals, except we use $r$ and $\theta$ instead of $x$ and $y$. As you will see in the coming couple of weeks, the biggest challenge with computing triple integrals is determining what the bounds of integration are. However, we do not have to worry about that for now, since the bounds have already been set for you.
\begin{align*}
\int_0^{2 \pi} \int_0^1 \int_{-1/2}^{1/2} (r^2 \sin^2 \theta + z^2) dz r dr d\theta &= \int_0^{2 \pi} \int_0^1 (r^3 \sin^2 \theta + \frac{1}{12}r)dr d\theta \\
&= \int_0^{2 \pi} \left( \frac{1}{4}\sin^2 \theta + \frac{1}{24}\right) d\theta\\
&= \frac{\pi}{3} \quad\blacksquare
\end{align*}


\end{document}

